% Document Class
\documentclass[11pt, a4paper]{article}

%----------------------------------------------------------------
%                           PACKAGES
%----------------------------------------------------------------

\usepackage[utf8]{inputenc}
\usepackage[T1]{fontenc}
\usepackage{amsmath, amssymb, amsfonts} % Essential math packages
\usepackage[english]{babel}
\usepackage{graphicx}         % For including images
\usepackage[a4paper, margin=1in]{geometry} % Set page margins
\usepackage[dvipsnames, svgnames]{xcolor}   % For custom colors
\usepackage{fancyhdr}         % For custom headers and footers
\usepackage{lipsum}           % For dummy text (you can remove this)

% Hyperlinks package - should generally be loaded last
\usepackage{hyperref}
\hypersetup{
    colorlinks=true,
    linkcolor=RoyalBlue,
    filecolor=magenta,      
    urlcolor=teal,
    citecolor=green,
    pdftitle={Class Notes},
    pdfpagemode=FullScreen,
}

% Package for creating styled boxes (for definitions, theorems, etc.)
\usepackage[most]{tcolorbox}

%----------------------------------------------------------------
%                        HEADER & FOOTER
%----------------------------------------------------------------

\pagestyle{fancy}
\fancyhf{} % Clear all header and footer fields
\lhead{\courseCode: \courseTitle \\ \professorName, \institution} % Course title on the left of the header
\rhead{\lectureDate} % Lecture date on the right of the header
\cfoot{\thepage}     % Page number in the center of the footer

%----------------------------------------------------------------
%                        TITLE INFORMATION
%----------------------------------------------------------------

% --- Define commands for title elements ---
\newcommand{\courseTitle}{Math Method of Physics}
\newcommand{\courseCode}{PH205}
\newcommand{\professorName}{Justin David}
\newcommand{\institution}{IISc}
\newcommand{\lectureNumber}{XX}
\newcommand{\lectureTopic}{Default Lecture Topic}
\newcommand{\lectureDate}{\today}

% --- Redefine the maketitle command for a custom look ---
\makeatletter
\renewcommand{\maketitle}{
    \begin{center}
        \vspace*{1cm}
        {\Huge\bfseries \courseTitle}
        \vspace{0.5cm}
        
        \hrule
        \vspace{0.5cm}
        
        {\Large \textbf{Lecture \lectureNumber:} \lectureTopic}
        \vspace{0.5cm}
        
        \begin{tabular}{ll}
            \bfseries Instructor: & \instructorName \\
            \bfseries Date: & \lectureDate \\
        \end{tabular}
        
        \vspace{0.5cm}
        \hrule
        \vspace{1cm}
    \end{center}
}
\makeatother

%----------------------------------------------------------------
%                   CUSTOM NOTE ENVIRONMENTS
%----------------------------------------------------------------

% --- Definition Box ---
\newtcolorbox{definition}[1]{
    colback=LightSkyBlue!10,
    colframe=RoyalBlue,
    fonttitle=\bfseries,
    title=Definition: #1
}

% --- Theorem Box ---
\newtcolorbox{theorem}[1]{
    colback=Yellow!10,
    colframe=Goldenrod,
    fonttitle=\bfseries,
    title=Theorem: #1
}

% --- Example Box ---
\newtcolorbox{example}{
    colback=SeaGreen!10,
    colframe=SeaGreen,
    fonttitle=\bfseries,
    title=Example
}

% --- Key Point/Remark Box ---
\newtcolorbox{keypoint}{
    colback=Red!5,
    colframe=Red!75!black,
    fonttitle=\bfseries,
    title=Key Point!
}


%%%%%%%%%%%%%%%%%%%%%%%%%%%%%%%%%%%%%%%%%%%%%%%%%%%%%%%%%%%%
%                     DOCUMENT STARTS HERE                   %
%%%%%%%%%%%%%%%%%%%%%%%%%%%%%%%%%%%%%%%%%%%%%%%%%%%%%%%%%%%%

%----------- SET YOUR LECTURE DETAILS HERE -----------
\renewcommand{\courseTitle}{PH205: Math Method of Physics}
\renewcommand{\lectureNumber}{2}
\renewcommand{\lectureTopic}{Math Method of Physics}
\newcommand{\instructorName}{Justin David}
\renewcommand{\lectureDate}{August 4, 2025}
\renewcommand{\instructorName}{Justin David}
%-----------------------------------------------------

\begin{document}

\maketitle % This command generates the title using the details above

\tableofcontents % Creates a table of contents from your sections
\newpage

%----------------------------------------------------------------
%                   START WRITING YOUR NOTES
%----------------------------------------------------------------

\section{Newton's First Law: The Law of Inertia}

Newton's first law states that an object will remain at rest or in uniform motion in a straight line unless acted upon by an external force. This property of an object to resist changes in its state of motion is called \textbf{inertia}.

\begin{keypoint}
    An object's velocity, $\vec{v}$, is constant if and only if the net force, $\sum \vec{F}$, acting on it is zero. Mathematically:
    $$ \sum \vec{F} = 0 \iff \frac{d\vec{v}}{dt} = 0 $$
\end{keypoint}

\section{Newton's Second Law: Force and Acceleration}

This is the most famous of the three laws. It provides a quantitative relationship between force, mass, and acceleration.

\begin{definition}{Force}
    In physics, a force is an influence that can change the motion of an object. A force can cause an object with mass to change its velocity (e.g., moving from a state of rest), i.e., to accelerate.
\end{definition}

The law is expressed by the formula:
$$ \vec{F}_{net} = m\vec{a} $$
Where:
\begin{itemize}
    \item $\vec{F}_{net}$ is the net force vector.
    \item $m$ is the mass of the object (a scalar).
    \item $\vec{a}$ is the acceleration vector.
\end{itemize}

\begin{example}
    A 10 kg box is pushed on a frictionless surface with a horizontal force of 50 N. What is its acceleration?
    \begin{center}
        % For images, make sure the image file (e.g., box_diagram.png)
        % is in the same directory as your .tex file.
        % \includegraphics[width=0.4\textwidth]{box_diagram.png}
    \end{center}
    
    Using Newton's second law:
    $$ a = \frac{F}{m} = \frac{50 \text{ N}}{10 \text{ kg}} = 5 \text{ m/s}^2 $$
\end{example}

\section{Newton's Third Law: Action and Reaction}

For every action, there is an equal and opposite reaction.

\begin{theorem}{Action-Reaction Pairs}
    If object A exerts a force $\vec{F}_{AB}$ on object B, then object B simultaneously exerts a force $\vec{F}_{BA}$ on object A, and the two forces are equal in magnitude and opposite in direction:
    $$ \vec{F}_{AB} = -\vec{F}_{BA} $$
\end{theorem}

\lipsum[1-2] % This is just dummy text, you can delete it.

\end{document}% --- Problem Box ---
\newtcolorbox{problem}[1]{
    colback=Thistle!10,
    colframe=Plum,
    fonttitle=\bfseries,
    title=Problem: #1
}

% --- Solution Box ---
\newtcolorbox{solution}{
    colback=Honeydew!10,
    colframe=ForestGreen,
    fonttitle=\bfseries,
    title=Solution
}
