\documentclass[11pt, a4paper]{report}

%----------------------------------------------------------------
%                           PACKAGES
%----------------------------------------------------------------

\usepackage[utf8]{inputenc}
\usepackage[T1]{fontenc}
\usepackage{amsmath, amssymb, amsfonts} % Essential math packages
\usepackage[english]{babel}
\usepackage{graphicx}         % For including images
\usepackage[a4paper, margin=1in]{geometry} % Set page margins
\usepackage[dvipsnames, svgnames]{xcolor}   % For custom colors
\usepackage{fancyhdr}         % For custom headers and footers
\usepackage{lipsum}           % For dummy text (you can remove this)

% Hyperlinks package - should generally be loaded last
\usepackage{hyperref}
\hypersetup{
    colorlinks=true,
    linkcolor=RoyalBlue,
    filecolor=magenta,      
    urlcolor=teal,
    citecolor=green,
    pdftitle={Class Notes},
    pdfpagemode=FullScreen,
}

% Package for creating styled boxes (for definitions, theorems, etc.)
\usepackage[most]{tcolorbox}

%----------------------------------------------------------------
%                        HEADER & FOOTER
%----------------------------------------------------------------

\pagestyle{fancy}
\fancyhf{} % Clear all header and footer fields
\lhead{\courseTitle} % Course title on the left of the header
\rhead{\lectureDate} % Lecture date on the right of the header
\cfoot{\thepage}     % Page number in the center of the footer

%----------------------------------------------------------------
%                        TITLE INFORMATION
%----------------------------------------------------------------

% --- Define commands for title elements ---
\newcommand{\courseTitle}{Default Course Title}
\newcommand{\lectureNumber}{XX}
\newcommand{\lectureTopic}{Default Lecture Topic}
\newcommand{\lectureDate}{\today}
\newcommand{\instructorName}{Professor Name}

% --- Redefine the maketitle command for a custom look ---
\makeatletter
\renewcommand{\maketitle}{
    \begin{center}
        \vspace*{\fill}
        {\Huge\bfseries \courseTitle}
        \vspace{0.5cm}
        
        \hrule
        \vspace{0.5cm}
        
        {\Large \textbf{Lecture \lectureNumber:} \lectureTopic}
        \vspace{0.5cm}
        
        \begin{tabular}{ll}
            \bfseries Instructor: & \instructorName \\
            \bfseries Date: & \lectureDate \\
        \end{tabular}
        
        \vspace{0.5cm}
        \hrule
        \vspace*{\fill}
    \end{center}
}
\makeatother

%----------------------------------------------------------------
%                   CUSTOM NOTE ENVIRONMENTS
%----------------------------------------------------------------

% --- Definition Box ---
\newtcolorbox{definition}[1]{
    colback=LightSkyBlue!10,
    colframe=RoyalBlue,
    fonttitle=\bfseries,
    title=Definition: #1
}

% --- Theorem Box ---
\newtcolorbox{theorem}[1]{
    colback=Yellow!10,
    colframe=Goldenrod,
    fonttitle=\bfseries,
    title=Theorem: #1
}

% --- Example Box ---
\newtcolorbox{example}{
    colback=SeaGreen!10,
    colframe=SeaGreen,
    fonttitle=\bfseries,
    title=Example
}

% --- Key Point/Remark Box ---
\newtcolorbox{keypoint}{
    colback=Red!5,
    colframe=Red!75!black,
    fonttitle=\bfseries,
    title=Key Point! 📝
}


%%%%%%%%%%%%%%%%%%%%%%%%%%%%%%%%%%%%%%%%%%%%%%%%%%%%%%%%%%%%
%                     DOCUMENT STARTS HERE                   %
%%%%%%%%%%%%%%%%%%%%%%%%%%%%%%%%%%%%%%%%%%%%%%%%%%%%%%%%%%%%

%----------- SET YOUR LECTURE DETAILS HERE -----------
\renewcommand{\courseTitle}{PH205: Math Method of Physics}
\renewcommand{\lectureNumber}{}
\renewcommand{\lectureTopic}{Math Method of Physics}
\renewcommand{\lectureDate}{August 4, 2025}
\renewcommand{\instructorName}{Justin David}
%-----------------------------------------------------

\begin{document}

\maketitle % This command generates the title using the details above

\tableofcontents % Creates a table of contents from your sections
\newpage

\chapter{Prerequisites}

\section{Linear Operators}
A linear operator is a function between two vector spaces that preserves the operations of vector addition and scalar multiplication.
$$
A(e_i) = \sum_{j=1}^n A_{ji} e_j
$$
A linear operator maps from one vector space to another.
$$
V \to V'
$$
$$
e_i \to v'
$$
$$
i = 1 \dots n
$$
$$
j = 1 \dots n
$$
$$
\phi(e_i) = \sum_{j=1}^n A_{ji} f_j
$$
Example: map from $n$-dimensional vector space to $n$-dimensional vector space.
$$
\phi(e_i) = e_i
$$
$$
\phi(e_j) = 0, \quad j \ne i
$$
This results in a matrix:
$$
\begin{pmatrix}
1 & 0 & \dots \\
0 & 0 & \dots \\
\vdots & \vdots & \ddots
\end{pmatrix}
$$

\subsection{Projection Operator}
A projection operator is a linear operator $P$ from a vector space to itself such that $P^2 = P$.
$$
V = \sum_{i} c_i e_i
$$
$$
\phi^2(v) = \phi(\phi(v)) = \phi(\phi(c_i e_i)) = \phi(c_i e_i) = c_i e_i = v
$$
$$
\phi(c_i v_i) = v_i, \phi(c_i v_i) = v_i
$$
Example: mapping 2-dimensional to 3-dimensional v-s.
$$
\phi(e_1) = f_1+f_2+f_3
$$
$$
\phi(e_2) = f_1+f_2-f_3
$$
This gives the matrix representation $\phi$:
$$
\begin{pmatrix}
1 & 1 \\
1 & 1 \\
1 & -1
\end{pmatrix}
$$
Matrices occur naturally in representation of linear operator. Matrix algebra are inherited from the rules of linear algebra.
$$
A(c_i) =  A_{ji} c_j
$$
$$
B(c_i) = B_{ji} c_j
$$
$$
(A+B)(e_i) = A(e_i)+B(e_i) = A_{ji}e_j + B_{ji}e_j = (A_{ji}+B_{ji})e_j
$$

\section{Matrix Properties and Operations}

\subsection{Scalar Multiplication and Related Concepts}
Scalar multiplication is one of the basic operations defining a vector space.
$$
(\lambda A)(e_i) = \lambda A(e_i) = \lambda A_{ji} e_j
$$
$$
(\lambda A)_{ij} = \lambda A_{ij}
$$
$$
(A^{-1})_{ij} \cdot (A^{-1})_{ij} = \delta_{ij}
$$
$A^{-1}$ has to be a square matrix.

$$
e^A = I + A + \frac{A^2}{2!} + \dots
$$
Similarly there is $\sin(A)$, $\cos(A)$, $\cosh(A)$, $\sinh(A)$.
$$
(1-A)^{-1} = \frac{1}{1-A} = I + A + A^2 + \dots
$$
Method of computing inverse for matrices close to identity.

\subsection{Matrices related to a matrix $A$}
$$
(A^T)_{ij} = A_{ji} \quad \text{(Transpose)}
$$
$$
(A^\ast)_{ij} = (A_{ij})^\ast \quad \text{(Complex conjugate)}
$$
$$
(A^\dagger)_{ij} = (A^T_{ij})^\ast = (A_{ji})^\ast \quad \text{(Hermitian conjugate)}
$$

\subsection{Special Matrices}
\begin{itemize}
    \item $A^T=A$: Symmetric
    \item $A^T=-A$: Anti-symmetric
    \item $A^\dagger=A$: Hermitian
    \item $A^\dagger=-A$: Anti-Hermitian
    \item $A^\ast=A$: Real
\end{itemize}

\section{Advanced Matrix Concepts}

\subsection{Diagonal and Block Diagonal Matrices}
\begin{itemize}
    \item Diagonal matrix $A_{ij}=0$ if $i \ne j$.
    \item Upper triangular matrix $A_{ij}=0$ if $i>j$.
\end{itemize}
Matrices can be decomposed into block diagonal.

\subsubsection{Block diagonal matrix}
A \textbf{block diagonal matrix} is a square matrix that is partitioned into smaller square matrices (blocks) along its diagonal, with all off-diagonal blocks being zero matrices. Formally, a block diagonal matrix $M$ can be written as:
$$
M = \begin{pmatrix}
B_1 & 0 & \cdots & 0 \\
0 & B_2 & \cdots & 0 \\
\vdots & \vdots & \ddots & \vdots \\
0 & 0 & \cdots & B_k
\end{pmatrix}
$$
where each $B_i$ is a square matrix (block) of size $n_i \times n_i$ and the zeros represent zero matrices of appropriate sizes.

\textbf{Properties:}
\begin{itemize}
    \item The determinant of a block diagonal matrix is the product of the determinants of its blocks:
    $$
    \det(M) = \prod_{i=1}^k \det(B_i)
    $$
    \item The trace of a block diagonal matrix is the sum of the traces of its blocks:
    $$
    \text{Tr}(M) = \sum_{i=1}^k \text{Tr}(B_i)
    $$
    \item The inverse of a block diagonal matrix (if all blocks are invertible) is the block diagonal matrix of the inverses:
    $$
    M^{-1} = \begin{pmatrix}
    B_1^{-1} & 0 & \cdots & 0 \\
    0 & B_2^{-1} & \cdots & 0 \\
    \vdots & \vdots & \ddots & \vdots \\
    0 & 0 & \cdots & B_k^{-1}
    \end{pmatrix}
    $$
\end{itemize}

\textbf{Example:}
$$
\begin{pmatrix}
1 & 0 & 0 & 0 \\
0 & 2 & 0 & 0 \\
0 & 0 & 3 & 4 \\
0 & 0 & 5 & 6
\end{pmatrix}
=\begin{pmatrix}
\begin{matrix}1 & 0 \\ 0 & 2\end{matrix} & 0 \\
0 & \begin{matrix}3 & 4 \\ 5 & 6\end{matrix}
\end{pmatrix}
$$
Here, the matrix is block diagonal with two blocks: a $2\times2$ diagonal block and a $2\times2$ non-diagonal block.

Block diagonal matrices are useful in linear algebra because they allow us to break down complex problems into smaller, more manageable subproblems, especially when dealing with direct sums of vector spaces or simplifying linear transformations that act independently on subspaces.

\subsection{Trace and Determinant}
Trace of a matrix is sum of diagonal.
$$
\text{Tr}(A) = \sum A_{ii}
$$
\subsubsection{How to find the trace of a matrix}
$$
a_{11}x_1+a_{12}x_2 = c_1
$$
$$
a_{21}x_1+a_{22}x_2 = c_2
$$
$$
\begin{pmatrix}
a_{11} & a_{12} \\
a_{21} & a_{22}
\end{pmatrix}
\begin{pmatrix}
x_1 \\ x_2
\end{pmatrix}
=
\begin{pmatrix}
c_1 \\ c_2
\end{pmatrix}
$$
$$
x_1 = \frac{
\begin{vmatrix}
c_1 & a_{12} \\
c_2 & a_{22}
\end{vmatrix}
}{
\begin{vmatrix}
a_{11} & a_{12} \\
a_{21} & a_{22}
\end{vmatrix}
}
$$
\subsubsection{Determinant of a matrix}
$$
\det A =
\begin{vmatrix}
a_{11} & a_{12} & \dots & a_{1n} \\
a_{21} & a_{22} & \dots & a_{2n} \\
\vdots & \vdots & \ddots & \vdots \\
a_{n1} & a_{n2} & \dots & a_{nn}
\end{vmatrix}
$$
$$
\epsilon^{123\dots n} = 1
$$
$$
\epsilon^{2134\dots n} = -1
$$
$$
\epsilon^{i_1 i_2 \dots i_n} = 0
$$
$$
\det A = \sum_{i_1, i_2, \dots, i_n} \epsilon^{i_1 i_2 \dots i_n} a_{1 i_1} a_{2 i_2} \dots a_{n i_n}
$$

\subsection{Applications in Quantum Mechanics}
\subsubsection{Considering 2 particle fermions}
The wave function needs to be anti-symmetric.
$$
\psi(x_1) \psi(x_2) - \psi(x_2) \psi(x_1)
$$

Slater determinant
$$
\begin{vmatrix}
\psi_1(x_1) & \psi_2(x_1) & \dots & \psi_n(x_1) \\
\psi_1(x_2) & \psi_2(x_2) & \dots & \psi_n(x_2) \\
\vdots \\
\psi_1(x_n) & \psi_2(x_n) & \dots & \psi_n(x_n)
\end{vmatrix}
=
\begin{vmatrix}
\psi_1(x_1) & \psi_2(x_1) \\
\psi_1(x_2) & \psi_2(x_2)
\end{vmatrix}
\begin{vmatrix}
\psi_1(x_1) & \psi_2(x_1) \\
\psi_2(x_1) & \psi_1(x_1)
\end{vmatrix}
$$

\subsection{Matrix Inversion}
\subsubsection{Inverse of a Matrix}
$$
(A^{-1})_{ij} = \frac{\text{Cofactor } A_{ji}}{|A|}
$$
Cofactor $A_{ij} = (-1)^{i+j}$

\subsubsection{Gauss-Jordan Matrix Inversion}
\begin{itemize}
    \item $A_{ij} \to \lambda A_{ij} \quad \forall j$ (Row operation: multiplication by $\lambda$)
    \item Subtract a row by a multiple of another.
    \item $A_{ij} \to A_{ij}$
\end{itemize}

\end{document}
